\chapter{Introduction}
\label{chap:introduction}

\section{Motivation}
\label{sec:motivation}

The Science and Technology Entry Program (STEP), is a pre-collegiate preparation program under the New York State Education Department (NYSED). Its purpose is to help prepare minorities, historically underrepresented, or economically disadvantaged secondary school students for entry into postsecondary degree programs in scientific, technical, health-related, and the licensed professions \cite{nysed-step-website}. Recently, STEP released a request for proposals in which it outlined a set of program priorities. An institute looking to receive funding through STEP would be given a high precedence should they propose to provide one or more of those program priorities. Paraphrased, these priorities are for institutions to provide programs and services to improve:
\begin{enumerate}[\indent P1.]
	\item The recruitment and retention of male participants;
	\item The recruitment and retention of Hispanic/Latino and American Indian participants;
	\item Eighth grade students' NYS Math and Science Assessment examination scores \cite{nys-step-op-manual}.
\end{enumerate}

Additionally, STEP asserts a set of service and data requirements which its participating institutions must provide evidence of accomplishing in an annual report to continue to receive funding. These requirements are as follows:
\begin{enumerate}[\indent R1.]
	\item Provide evidence of formal collaborations between the funded institute, local school districts, collegiate outreach programs, and professional agencies;
	\item Provide program activities to assist students in acquiring the skills and aptitudes necessary to pursue postsecondary education leading to STEM (Science, Technology, Engineering, and Mathematics) fields;
	\item Provide services to enhance and increase students' involvement in research, internships, college level coursework;
	\item Provide program services to enhance students� mathematical and scientific skills in accordance with the Advanced Regents Diploma; and
	\item Develop a means of involving a student's parents and give them and clearly define their relationship with the funded institution's program \cite{nys-step-op-manual}.
\end{enumerate}

Data published in numerous studies \cite{timms-95, before-its-too-late} show that American 8th grade students perform consistently below those in many other countries around the world. School Districts in Northern New York can substantiate the STEP priorities and requirements. NYS Mathematics test data from 2010 revealed that 56\% of the school districts in St. Lawrence County had below average proficiency levels for 8th grade \cite{step-application-11}. At the Salmon River School District, where there is a majority American Indian demographic, 54\% of students scored below acceptable proficiency levels on the 8th Mathematics assessment \cite{nny-prism}. It is clear that STEP identified these troubling facts and is now pushing institutes of higher education to try to solve these issues.

IMPETUS (Integrated Mathematics and Physics for Entry To Undergraduate STEM), for Career Success is a STEP program resulting from the collaboration between St.Lawrence-Lewis County BOCES and Clarkson University. Its primary goal is to improve and increase opportunities for underrepresented minorities and students from economically disadvantaged rural areas to realize their potential for college entry as STEM majors and for eventual career success in technically oriented professions. IMPETUS satisfies their goal and addresses all of the STEP priorities/requirements by providing a collection of activities, programs, and services during both the academic year and summer to its participants. During the academic year, workshops are organized to occur on Clarkson's campus that are designed to improve student' academic skills and awareness of career paths. During the summer, students participate in a roller coaster design competition while paired with mentors that are college students. This competition acts as a framework for learning and applying math and science skills.

Traditionally, collaboration in IMPETUS has been accomplished via in-person meetings between various combinations of its relevant parties. St. Lawrence County is the largest county by area in not only Northern New York, but in the whole state, yet has a population density of only $41/\mathrm{mi.}^2$, whereas the state average is $355/\mathrm{mi.}^2$. The region of Northern New York constitutes nearly 20\% of the state's land area, yet only 2\% of its population and a mere 0.2\% of the state's received federal funding. In a region of these attributes, it should not be surprising that the presence of traditional STEM learning venues such as museums, aquariums, science centers, and high-tech industries are basically non-existent. Combined, these circumstances do not create an environment suited towards an efficient and effective exchange of information \cite{nny-prism}. As it stands now, collaboration between all IMPETUS participants only occur seven times per academic year, once for each on-campus event.

IMPETUS has proposed the creation of a web-based Collaborative Environment that aid in satisfying the STEP priorities and requirements by combining different technologies into a unified solution while increasing the convenience and accessibility of collaboration. Students who are in need of assistance, when meeting in-person is not possible, will be able to engage in personalized web-based cooperative learning (i.e. tutoring and mentoring) with Clarkson University students and web-based self-learning all using state-standard based materials and assessments. Educators will be able to survey and receive feedback from targets regardless of their location and parents will have access to their children's data and program information. With this system, collaboration between all IMPETUS participants would not hindered by the geographical properties of Northern New York. Instead, collaboration can occur at any place with a computer and an internet connection.

This effort is in alignment with a national need for computer literacy in education. In 2008, the National Science Foundation (NSF) created a Cyberlearning Task Force that was charged with, amongst other objectives, determining what the key research topics were surrounding cyberlearning \cite{nsf-taskforce}. In a subsequent NSF publication, it was determined that one of these topics should be collecting, analyzing, and managing data about how individuals use a given cyberlearning system \cite{nsf-cyberlearning}. Due to STEP's data logging requirement and the fact that the NSF has deemed cyberlearning data collection to be an area of research interest, the IMPETUS Collaborative Environment will collect an extensive amount of data about not only the individuals using it, but also how they are using it.

\section{Related Work}
\label{sec:related-work}

The proposed IMPETUS Collaborative Environment (hereafter referred to as simply the ``Collaborative Environment'') can be categorized as a Cyberlearning System (CLS). There are two classes of existing CLSs: the traditional, such as WebCT \cite{webct} (which has since been bought and renamed by Blackboard, Inc. \cite{blackboard-dot-com}) and Moodle \cite{moodle-dot-net}, and the specialized, such as Khan Academy \cite{khanacademy-dot-org}. 

Traditional CLSs provide access to educational resources in addition to user and course management features. Teachers can manage multiple courses, make documents available to their students, create quizzes for the students to take, and are overall very robust pieces of software, but this does not mean they are without their drawbacks. WebCT is a proprietary and licensed piece of software meaning that one is not able to create novel features that interact with the existing system, but rather one only has access to what Blackboard, Inc. provides. Moodle, being free and open source software, does allow for the creation of novel features in the form of plug-ins. Moodle is capable of providing some necessary functionality, such as user management, quizzes, and surveys, but does not offer specific data analysis and web tutoring/mentoring functionality. Additionally, traditional CLSs, because they are meant to manage an entire school, provide a substantial amount of unnecessary features and complexity. Ultimately, they provide a lot of features that are not needed, some features that are necessary, and none of others that are necessary.

A specialized CLS is a provider of educational resources and user management features. Khan Academy provides people with educational videos and associated exercises. If someone wants to track their exercise scores or details about the videos they have watched, they must register for an account on the Khan Academy website. While this is perhaps the easiest CLS to start using, it does not allow for any flexibility, just as the proprietary WebCT, and does not provide all of the essential features that the Collaborative Environment needs.

The Khan Academy website is entirely open source, including their exercise framework. During the planning for the Collaborative Environment we initially reasoned that we could use this exercise framework as a drop-in solution for allowing the creation of quizzes in the Collaborative Environment, but upon further investigation creating an exercise was not a task that could be accomplished by any type of user. Creating exercises using the Khan Academy software requires extensive use of HTML and JavaScript. In order to meet our objective of having a system that is accessible to many types of users, a technical wall such as this is unsatisfactory. This notion was the primary motivating factor behind the creation of WebCT, that all users should be able to utilize web-based learning environments, not just those with a technical background \cite{World Wide Web - Course tool}.

In terms of a CLSs being relevant research tools, many papers have been written that offer conclusions indicating that the future of education will almost definitely include online components \cite{keeping-pace2-pdf} and that, on average, students perform equally or better academically with online learning as with in-class learning \cite{National Primer on K-12 Online Learning.pdf}. But, they follow with the notion that these ``conclusions'' are in need of more confidence via further research \cite{National Primer on K-12 Online Learning.pdf}.

To summarize, the factors motivating the creation of a new CLS, the Collaborative Environment, are that it needs to free and open source, accessible by users of technical and non-technical backgrounds, and provide a set of functionality not wholly available in an existing solution. Additionally, because it is evident that there is a need for further CLS research \cite{National Primer on K-12 Online Learning.pdf, National Primer on K-12 Online Learning.pdf}, the data-logging objective of the Collaborative Environment makes it a prime candidate.

\section{Methodology}
\label{sec:methodology}

Creating the Collaborative Environment will be done following a modified waterfall model of software engineering: Requirements, Design, Implementation, and Verification \cite{stober2010agile}; the modification being that feedback will be readily incorporated into the phases allowing progress to go ``up" the waterfall. Being that there is only one developer on the the project, it is the most basic model to follow.

As modules become ready for verification, usability tests will be done to get feedback on their usefulness and design. This will require the creation of surveys and the user studies gauge how effective the interface is at allowing a user to accomplish goals.

\section{Contributions}
\label{sec:objectives}

The primary contribution of this work is a free and open source\footnote{Download the source code at: \url{http://link-to-program.com}} web-based Collaborative Environment that will serve as
\begin{inparaenum}[\itshape 1\upshape)]
	\item a solution to both overcoming the challenges in operating a STEM outreach program in geographically isolated, rural areas; and
	\item a tool for further research into web-based (cyber) learning environments.
\end{inparaenum} The design of this software iteratively incorporates usability research performed using multiple different types of expected users of the system into its design and implementation.

\section{Thesis Summary}
This thesis is outlined as follows: Chapter 2 provides an overview of the Collaborative Environment. Defined are the goals of the system (i.e. what tasks should be accomplishable by using it), the primary features that will realize the goals, and what the perceived classes of users of the system will be. Chapter 3 provides a background of the preexisting software and services that were utilized in its development and continues into the detailed explanations of each primary feature's concept, design, and implementation. Chapter 4 details the usability study that was performed to gather an initial set of feedback on the Collaborative Environment. The study was accomplished with a user survey and described are its goals, participants, testing procedure, responses, and results. Chapter 5 discusses the conclusions of this work and where it can lead in the future.

