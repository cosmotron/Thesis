\chapter{Software Overview}
\label{chap:software-overview}

\section{Goals}
\label{sec:overview-goals}

The primary motivations behind the creation of the Collaborative Environment are the need for increased collaboration and the need for data logging. The Collaborative Environment will be targeted at a variety of different user types (user classes), it follows that each will be permitted access to a different set of features and data. This means, that while an increase in collaboration is an overall goal for all user classes, specific collaborative goals vary based on how much access each class has:

\begin{itemize}
	\item All users should have read access to announcements and an event calendar.
	\item All users should have access to an internal messaging system for one-to-one and one-to-many communication.
	\item All users should be able to answer surveys that are available to them.
	\item Students should be able to answer quizzes that are available to them.
	\item Students should have access to an availability schedule for teaching assistants.
	\item Students should have access to external educational resources.
	\item Teaching assistants should be able to create their own availability schedule.
	\item Teaching assistants should be able to create quizzes and surveys.
	\item Teachers should be able to create announcements and events for the calendar.
\end{itemize}

The next set of goals correspond to data logging. There is the possibility of some conceptual overlap between whether a goal is collaborative or data logging related, so the line has been drawn at whether or not the collaborative aspect is simply incidental to having to log data. If that is the case, then the goal will categorized under data logging.

\begin{itemize}
\item Parents should have access to their children's data.
\item Students should have access to their own data.
\item Teaching assistants should have access to the students that they are assisting.
\item Teachers should be able to be able to view reports for the students that are available to them.
\item Teachers should be able to edit data for the students that are available to them.
\item Administrators should be able to create and access the entities by which data is stored.
\item Administrators should be able to access user tracking data.
\end{itemize}

In the above list, there are two terms which require definition: \emph{data} and \emph{entities}. \emph{Data} is defined as all personal information, survey results, and quiz results for the user it is in reference to, whereas \emph{entities} are the fundamental objects that provide the basis for all relationships in the user data.


\section{Features}
\label{sec:overview-features}
The features of the Collaborative Environment are designed and implemented to accomplish the goals outlined in \S \ref{sec:overview-goals}. The following provide a high-level overview of each of the major features of the Collaborative Environment and describe which goals they aid in accomplishing. More specific details regarding each can be found in \S \ref{sec:design-features}.

\begin{description}
	\item [Academic Year] \hfill \\ The Collaborative Environment will be used for multiple academic years, each of which have their own data that is dependent on it. For example, different quizzes might be given each year and students are usually in a different grade and classes. To give users access to a previous year's data, there is a simple interface to switch between them on-the-fly. This is a necessity for a robust data logging tool.
	\item [User Management] \hfill \\ In order to keep track of data about individuals, each must have an account. Users that a registered as students will have data such as their gender and ethnicity, standardized test performance, and quiz results stored in this system.
	\item [District Management] \hfill \\ Teachers, teaching assistants, and students are all associated with a district to better organize their data for analysis and to control access to private student data.
	\item [Learning Pathway] \hfill \\ Students are provided with a visual representation of the knowledge that they should have as their school year progresses. This representation is a hierarchical tree of concepts based on both the New York State Core Content and Common Core State standards. Each node in this tree is an individual concept and has associated with it both video resources and quizzes. The tree acts as a guide to what students know and what they need more practice in. Students and their teaching assistants will be able to use these identified problem areas during collaborative-learning.
	\item [Quiz System] \hfill \\ A system by which teachers and teaching assistants can create assessments of their student�s knowledge is one of the key collaborative goals of this system. By allowing teachers and teaching assistants to create quizzes quickly and easily, it will allow them to get valuable feedback from their students. From the student�s perspective, they�ll have access to a means of studying and brushing up on topics that they might feel they are struggling with.
	\item [Survey System] \hfill \\ Teachers and teaching assistants can get additional feedback in a non-quiz like fashion. It can be used as an early intervention indicator in that if a student is not participating it might indicate that this student needs more attention. Feedback can also be garnered from student's parents thus increasing their involvement with IMPETUS.
	\item [Messaging] \hfill \\ To facilitate communication between all of these users, a messaging system was devised that would allow for a quick means of contacting either a user or a group of users. This system keeps track of messages internally, but is capable of notifying a user of a new message via email if they have an email address associated with their account.
	\item [Analytics] \hfill \\ An organization funded by STEP must submit, each a year, a report to the Education Department that summarizes data about the participants, activities, program content, and outcomes. The Collaborative Environment collects all of this data but it often needs manipulation and formatting to output the exact data that this annual report mandates. The analytical system is responsible for this task.
\end{description}

\section{User Classes}
\label{sec:overview-user-classes}

In the Collaborative Environment, there exists a hierarchy of roles where each user will be a member of a particular level. These roles naturally correspond to the type of user that they describe and are listed in order of least access to most (also note that any level's access to features is a superset of its preceding level's features). Further, there is a distinction between non-privileged and privileged classes: users of a non-privileged class have access to zero or one user's data (typically their own or their child's) whereas users of a privileged class have access to zero or more users data (typically all of their students).

\begin{description}
	\item [Non-privileged] \hfill
		\begin{description}
			\item [Anonymous] \hfill \\ An anonymous user is any user that has not presented any authentication credentials. They have read access to announcements, calendars, and schedules.
			\item [Parent] \hfill \\ Has access only to their student's educational data, quiz results, and attendance. They are able to participate in surveys and send messages to any user.
			\item [Student] \hfill \\ May participate in quizzes and view their own various attempts at quizzes.
		\end{description}
	\item [Privileged] \hfill
		\begin{description}
			\item [Teaching Assistant] \hfill \\ For any district that they are enrolled as an assistant for, they may view any Student's data. They may also create surveys and quizzes.
			\item [Teacher] \hfill \\ May read and write student data for only those that belong to the same district as themselves. This data includes survey results, quiz results, and detailed student educational information. A Teacher may also enroll a new Teaching Assistant into their district and  create new Students and Parents.
			\item [Administrator] \hfill \\ Responsible for creating the fundamental Entities which all other users interact with and have access to all user data and reports. They may create new academic years and districts, as well as the specific student activities, courses, and exams that may be tracked. They may also create a user of any role and assign Teachers to their respective districts.
		\end{description}
\end{description}