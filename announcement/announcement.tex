\documentclass[letterpaper,12pt]{report}
\usepackage{fullpage}

\begin{document}
\thispagestyle{empty}
\begin{center}
\Large{Overcoming Geographic Isolation: The Design and Implementation of a Web-based Collaborative Learning Environment}

\vspace{.2in}
\normalsize{A thesis by}\\
\large{Ryan Lewis}\\
\normalsize{Department of Computer Science}

\vspace{.2in}
\normalsize{December 15th, 2011\\
TAC 208 at 11:00am\\ 
\ \\
Examining Committee:\\
Dr. Christopher Lynch, advisor\\
Dr. Kathleen Fowler\\
Dr. David Wick}
\end{center}
\begin{center}
\textbf{Abstract}
\end{center}

IMPETUS for Career Success is a STEP (Science and Technology Entry Program) program resulting from the collaboration between St.Lawrence-Lewis County BOCES and Clarkson University. Its primary goal is to improve and increase opportunities for underrepresented minorities and students from economically disadvantaged rural areas to realize their potential for college entry as STEM (Science, Technology, Engineering, and Mathematics) majors and for eventual career success in technically oriented professions.

Traditionally, collaboration in IMPETUS between its participating students, teachers, assistants, and parents has been accomplished via in-person meetings. However, Northern New York's low population density and low federal financial assistance, despite constituting nearly 20\% of the state's land area, do not create an environment suited towards an efficient and effective exchange of information. This work describes a web-based Collaborative Environment designed to increase the convenience and accessibility of collaboration.

The described Collaborative Environment, which can be categorized as Cyberlearning System (CLS), is similar in nature to existing CLSs such as Blackboard, Moodle, and the Khan Academy, yet provides a novel combination of features: accessibility to both technical and non-technical users, personalized student self- and assisted-learning, and data logging for future analysis.

A usability study was performed via a survey distributed to current IMPETUS participants with the goals to verify the system design and gather an initial set of qualitative data for future quantification. Design changes left as comment feedback in the survey will lead to application changes which can then be quantitatively compared to the initial qualitative data.

\end{document}