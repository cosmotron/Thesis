\chapter{Software Design and Implementation}
\label{chap:software-design}

\section{Background}
This section will provide details of the existing software and services that are leveraged to create the Collaborative Environment. For each, we provide the reasoning for its use as well as a technical description.

\subsection{Symfony}
The Collaborative Environment is built using Symfony Standard Edition, an open-source, object-oriented PHP framework designed around the Model-View-Controller (MVC) software architecture. Using Symfony to create the Collaborative Environment encourages the use of design patterns that are well understood and allows for more of the development time to focus on application features rather than reimplementing standard components of web applications.

The MVC architecture separates an application's business logic, that is, all of the algorithms which process the exchange of data between an interface and a database, from its user interfaces. The \emph{model} consists of object representations of application data (Entities) and a persistence layer, which will store and retrieve entities via a databases management system. The \emph{view} will render a model object into a user interface, such as a web page. The \emph{controller} is what mediates transactions between the user and the application. The typical control flow of a basic MVC application is:

\begin{singlespacing}
\begin{enumerate}
	\item Client makes a request
	\item Controller receives the request and transforms it into a manipulation of the Model
	\item The updated Model is saved to the database by the persistence layer from the Controller
	\item A View is generated by the Controller which makes the changes to the Model visible
	\item The Controller responds to the Client's request with the generated View.
\end{enumerate}
\end{singlespacing}

Symfony Standard Edition comes with many software bundles which extend the core Symfony framework to provide an enterprise-grade application skeleton. Three of the key bundles included are the Security, Doctrine, and Twig bundles, which provide user-rights management, an object-relational mapper, and a template engine respectively.

\subsubsection{Security}
Security in Symfony is handled by a two-step process: authentication and authorization. The authentication step identifies a user and is typically accomplished by having a user first visit the website, at which point they are authenticated as an anonymous user, and then having them provide a set of credentials to authenticate them as a specific user. Authorization occurs when a user attempts to access a resource of an application. Symfony allows for a set of user roles to be defined, of which each user has a set of, as well as an access control list, to determine exactly what resources any given user has access to.

Controllers are aware of what the current user's authentication is and, as such, can be configured to only allow processing to occur if and only if the current user has a given role. While an Entity will generically describe all instances of a piece of data in an application, if the currently authenticated user should only have access to an exact instance of an entity, an entry can be added to an access control list for that particular user-entity instance pair.

These features provide the Collaborative Environment with a powerful and robust mechanism that is essential to securing the Student data being tracked.

\subsubsection{Doctrine}
In a dynamic web application, Entities are constantly being created, read, update, and deleted. To keep track of these changes, it is natural to consider using a database management system (DBMS) such as MySQL or PostgreSQL. In an object-oriented program, these Entities often contain non-scalar data (e.g. lists, arrays) that must be persisted by a DBMS, which can be problematic because a DBMS can often only store scalar data (e.g. strings, integers). Using an object-relational mapper (ORM), such as Doctrine, solves this problem by managing the transformation to and from scalars/non-scalars when persisting an object.

As an example of how an ORM operates, let's consider an object-oriented system where a Student is enrolled in a set of Courses. Naturally, we would have two Entity objects, Student and Course. The fields of the Student entity could be an ID, a name, and a list of Courses that they're enrolled in (consider this to be a many-to-many relationship). The fields of the Course Entity could be a name and number. When a new Student has been created and they are enrolled in, perhaps, two different Courses and the ORM is told to persist this data, it will store the scalar data in the  DBMS as its native types (names map to varchar, IDs and numbers map to integer), but for the Student's list of enrolled courses, since we established that there is a many-to-many relationship, the ORM will manage a Student ID-Course number tuple (Student 1 is enrolled in Course 1, Student 1 is enrolled in Course 2). Of course, when this data is requested from the DBMS, the ORM will convert the data back into Entity objects.

\subsubsection{Twig}
When a Controller has to return a new response, it is common for the View to be defined as a template that will be rendered though an engine, such as Twig, rather than explicitly crafted in the desired output format. For example, this could be accomplished by designing a Twig template that will output specific Model Entity variables and have it be processed into HTML by the Twig engine instead of just writing a mixture of HTML and PHP directly.

There are many benefits to using a template engine, such as:

\begin{description}
	\item [Template inheritance] Defining a parent or a layout template that can be inherited from children allowing for a consistent theme across an application.
	\item [Syntax] Provides numerous shortcuts for applying common patterns to variables such as escaping output and modifying control flow.
	\item [Speed] After compiling a template, the result will be optimized and low-overhead PHP when compared to freehand coding.
\end{description}

\subsection{Khan Academy}
The Khan Academy is a nonprofit organization that aims to better global education by providing free, high-quality resources to anyone anywhere. \cite{khan-website-about} On their website, they provide in excess of 2700 videos that teach varying topics in STEM fields. The organization has been provided with the financial support and public accolades of large entities such as The Bill \& Melinda Gates Foundation and Google \cite{khan-wiki-sources} indicating that they produce noteworthy content. The videos that Khan Academy hosts are all created by Salman Khan who holds an MBA from Harvard and three science degrees from MIT. Each is produced in a fairly consistent manner using a software whiteboard, a screen capture program, and a microphone while typically lasting about 10 minutes.

As stated in sections \ref{sec:overview-goals} and \ref{sec:overview-features}, students using the Collaborative Environment should have access to supplemental learning resources and will have the ability to take quizzes given by their teachers and teaching assistants. The Khan Academy videos are ideal for this use as they are concise and short enough to be associated with individual quizzes that are created in the Collaborative Environment. If a student, while taking a quiz, feels that they are in need of some assistance, a link to the an appropriate Khan Academy video will be readily available to them.

\section{Features}
\label{sec:features}

%%%% Academic
\subsection{Academic Year}
In a traditional document-based teaching environment, student data is stored in numerous grade books  or spreadsheets which are held by many different people in many different places. Each teacher, for instance, would have their own set of student records and a district will have a permanent record for each student containing information such as their standardized test scores.

As the Collaborative Environment 


\subsubsection{Backend Design}

\subsubsection{Frontend Design}

\subsubsection{Implementation}

%%%% Users
\subsection{User Management}

\subsubsection{Concept}

\subsubsection{Backend Design}

\subsubsection{Frontend Design}

\subsubsection{Implementation}

%%%% Districts
\subsection{District Management}

\subsubsection{Concept}

\subsubsection{Backend Design}

\subsubsection{Frontend Design}

\subsubsection{Implementation}

%%%% Quiz
\subsection{Quiz System}

\subsubsection{Concept}

\subsubsection{Backend Design}

\subsubsection{Frontend Design}

\subsubsection{Implementation}

%%%% Survey
\subsection{Survey System}

\subsubsection{Concept}

\subsubsection{Backend Design}

\subsubsection{Frontend Design}

\subsubsection{Implementation}

%%%% Messaging
\subsection{Messaging}

\subsubsection{Concept}

\subsubsection{Backend Design}

\subsubsection{Frontend Design}

\subsubsection{Implementation}

%%%% Analytics
\subsection{Analytics}

\subsubsection{Concept}

\subsubsection{Backend Design}

\subsubsection{Frontend Design}

\subsubsection{Implementation}

\section{Known Issues}