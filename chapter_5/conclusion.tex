\chapter{Conclusions}
\label{chap:conclusion}

\section{Conclusions}
The primary contribution of this work is a free and open source web-based Collaborative Environment that ultimately provides a means of satisfying the STEP priorities and requirements by way of aiding in overcoming the issues of operating a geographically isolated STEM outreach program. By basing the Collaborative Environment on state-determined criteria, it can be utilized by any STEP funded institution that has a desire to provide web-based collaborative learning to its participants. Additionally, its open source nature allows for local customization and adoption for state programs outside of New York which have goals similar to that of STEP's. It has been shown that the currently implemented Collaborative Environment's feature set addresses the STEP priorities and requirements. Specifically:

\begin{description}
	\item [User Management] \hfill \\ The storage of student data is facilitated by this system. It allows for personal data, such as gender and ethnicity, and educational data, such as standardized test scores, courses enrolled in, and after-school activities participated in, to be stored on a yearly basis. This data can be used to show improvement in the STEP program priorities: the recruitment/retention rates of male users and underrepresented minorities as well as tracking NYS Math and Science assessment examination scores.
	\item [Quiz System] \hfill \\ Allows coaches and teaching assistants to create quizzes to measure student performance and act as a supplemental educational tool during student collaborative and self-learning. Students can use these quizzes as a means of brushing up on topics they are struggling with, while teaching assistants can use quizzes as a way to ensure their students are learning. This system assists in the fulfillment of the STEP program requirements stating that there must be formal collaborations between the funded institute and the local school districts in addition to providing activities that will assist students in acquiring the skills and aptitudes necessary to pursue collegiate STEM disciplines.
	\item [Learning Pathway] \hfill \\ Using a visualization of state math standards in the form a hierarchical tree of dependencies, students can readily identify which topics they are proficient in and which they need assistance with. Each node in this tree represents a state standard and has associated with it both external video resources via the Khan Academy and quizzes created using the Quiz System. Nodes change color based on a student's performance on the quizzes so that a quick look at the tree will identify all the areas that an individual student does and does not need help with. This ability to know what students need help with on a individual and personalized basis via a tool backed by actual state standards will aid in accomplishing the STEP program requirement which states there must exist services to enhance the math and science skills in accordance with the Advanced Regent Diploma.
	\item [Survey System] \hfill \\ Coaches and teaching assistants can create surveys to be distributed to the Collaborative Environment's user base. This system can be used to gather valuable data and feedback from all of the participating bodies. Additionally, this can be used as a way to identify which students might need early intervention to keep them succeeding. If a user does not fill out a survey, it might be a sign that the student needs some additional attention.
	\item [Messaging] \hfill \\ Prior to the creation of this tool, there was no central resource for being able to communicate with any of the members of the IMPETUS program (students, coaches, assistants). Since all members will have an account (with an email address associated with it) in the Collaborative Environment, using a private message system which also notifies recipients that they have new mail is ideal for maintaining connections and collaboration.
\end{description}

Since the Collaborative Environment is open source and built using a set of well-defined existing tools, such as Symfony and Google Maps, it is a reasonably sustainable application for future work and maintenance. A future maintainer would have to be familiar with the concepts described in \S \ref{subsec:design-background-symfony}, SQL and database modeling, and object-oriented programming in PHP. Expected annual upkeep entails creating a new academic year, transitioning each district's roster to the newly created academic year, and ensuring that the standards-based learning pathway is accurate.

Additionally, a user survey was performed to gather some initial feedback regarding the Collaborative Environment. Although a lot of helpful comments were received, it is important to remember that the user study experiment was only run once and had a limited number of responses (18). Therefore, the qualitative responses that were received could not be used to concretely assert facts regarding the Collaborative Environment's usability. That being said, in an anecdotal sense, the results of this initial research have been primarily positive with 89\% of the participants indicating satisfaction with the system.

\section{Future Work}
There exist several areas of future work for this project:
\begin{itemize}
	\item The Analytics feature discussed in \S \ref{subsec:design-analytics} was designed to make completing the annual STEP summary report easier by formatting the data collected by the Collaborative Environment into that expected by the report. As of the current implementation, only 4 of the 16 data tables required by the annual report are generated.
	\item Conduct another usability study to obtain qualitative data regarding the redesigned features based on the first usability study. Once another usability test has been administered, it can be compared to the one presented in this paper to have quantitative data indicating, for instance, an improvement between designs. Future studies must involve IMPETUS tutors, mentors, and students as they were a missed in the initial one.
	\item Due to the Collaborative Environment's data logging capabilities, it can be used as a powerful research tool. For example, student quiz results for multiple academic years can be analyzed to determine which quiz questions resulted in different levels of student academic achievement.
	\item The open source nature of the Collaborative Environment allows it to be potentially utilized by other STEP programs. Making these other programs aware of its existence could lead to better software and, consequently, better STEP programs by building a community around the Collaborative Environment. Additionally, reaching out to the Khan Academy could lead to a mutually beneficial exchange of information regarding collaborative and educational software.
\end{itemize}